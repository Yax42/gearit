\begin{DoxyAuthor}{Authors}

\end{DoxyAuthor}
Epitech Innovative Project promotion 2015 ~\newline
 Group {\bfseries Gear\+It}\hypertarget{index_intro}{}\section{Introduction}\label{index_intro}
This package provides code templates for use by Gear\+It developers. ~\newline
 Farseer\+Physic engine is directly included in our souces for easy-\/access. ~\newline
 Each module is located via his module name \+: Robot, Map, Editor...

{\bfseries Gear\+It Links}
\begin{DoxyItemize}
\item \href{http://eip.epitech.eu/2015/gearit/}{\tt Website }
\item \href{https://www.facebook.com/pages/Gear-it/274074276087936}{\tt Facebook }
\item \href{https://github.com/Yax42/gearit/}{\tt Git\+Hub }
\end{DoxyItemize}

If using the code in this package as an example -\/ please modify the comments as appropriate for your own specific code.



 \hypertarget{index_requirements}{}\section{Requirements}\label{index_requirements}
Visual Studio 2010~\newline
 X\+N\+A 

 \hypertarget{index_todo}{}\section{Todo}\label{index_todo}
\begin{DoxyRefDesc}{Todo}
\item[\hyperlink{todo__todo000001}{Todo}]Robot textures/colors ~\newline
 Map textures/colors~\newline
 Lobby ~\newline
 Network sync with client physics ~\newline
 Ingame \hyperlink{namespace_g_u_i}{G\+U\+I} ~\newline
 Refont \hyperlink{namespace_g_u_i}{G\+U\+I} ~\newline
 Sound ~\newline
 \end{DoxyRefDesc}


 \hypertarget{index_notes}{}\section{Release notes}\label{index_notes}
{\bfseries Changelog-\/02072014}~\newline
 Games are effectivly scriptable in Lua now~\newline
 Paint colors on game objects~\newline
 You can now control the limits of rotation of your Revolute\+Spots in Lua for your robots~\newline
 Add triggers and artefacts in map editor~\newline
 Pieces of a same robots not colliding anymore~\newline
 Rod control W\+A\+Y more friendly in robot editor!~\newline
 Force and wheights correctly managed~\newline


{\bfseries Changelog-\/04032014}~\newline
 Application de textures sur les différents composants des robots.~\newline
 Protocole réseau en cours d’implémentation.~\newline
 Physique du jeu recorrigée.~\newline
 Mise en place d'une \hyperlink{namespace_g_u_i}{G\+U\+I} plus ergonomique.~\newline
 Mise en place du module de connexion sur le site web.~\newline
 Mise en place du module d'authentification sur le master server.~\newline
 